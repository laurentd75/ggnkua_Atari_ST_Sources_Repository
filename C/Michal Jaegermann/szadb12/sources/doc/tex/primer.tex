%%%%%%%%%%%%%%%%%%%%%%%
%   szadb primer      %
%   version 1.3       %
%   August, 1990      %
%%%%%%%%%%%%%%%%%%%%%%%
\documentstyle[twoside,11pt,verblist]{article}

\title{\protect{\LARGE{\tt szadb} {\bf primer}}}
\author{Micha{\l} Jaegermann \\[.6in]
	edited by\\
	Anthony Howe}
\date{13 Sept 1990}

%%
\newcommand{\szadb}{{\tt szadb}}
\newcommand{\adb}{{\tt adb.ttp}}
\newcommand{\name}[1]{{\tt #1}}
\newcommand{\key}[1]{{\sf $\langle$#1$\rangle$}}
\newcommand{\carret}{\char94}
\newlength{\exmpskip}
\setlength{\exmpskip}{2\parindent}
\newenvironment{exmpl}
  {\begin{quote}\setlength{\leftskip}{\parindent}}{\end{quote}}
\newcommand{\readexample}[1]
   {\medskip\inputverbatim[\exmpskip]{#1}\medskip\noindent}
%%
%\date{June 4, 1990}
\begin{document}
\vfill
\maketitle
\thispagestyle{empty}
%%
\titlepage\null\vfill{\sf 
\begin{quote}
	There is NO WARRANTY with respect to this publication, or the
	program it describes, and disclaim any implied or explicit
	suggestions of usefulness for any particular purpose.  Use
	this program only if you are willing to assume all risks, and
	damages, if any, arising as a result, even if caused by
	negligence or other fault.

\bigskip
	Sozobon C compiler and \szadb\ debugger are both copyright
	\copyright 1989 by Sozobon Ltd.  Both can be freely copied
	and distributed provided all copyright notices will remain
	intact and all modified versions will be clearly marked as such.

\medskip
	Atari, ST, TOS and GEMDOS are trademarks of Atari Co.\\
	GEM is a trademark of Digital Research Co.\\
	Mark Williams C is a trademark or Mark Williams Co.\\
	{\sc Unix} is a trademark of AT\&T.
\end{quote}
}\par\null\endtitlepage
\sloppy
%{\tt  % turn this on if you want to make life easy for dvi2tty
%\raggedright
\section{Introduction}

\szadb\ is an assembly level debugger for programs written for the TOS 
operating system on the Atari ST computer.
It is primarily meant to be a companion
to the freely distributed Sozobon C compiler, and can be copied and passed
around under the same conditions.

One of design goals of the original \szadb\ project was to provide 
a debugger closely resembling {\tt adb},
which is available on most {\sc Unix} systems. There are some dissimilarities
which follow mostly from differences in target machines
and operating systems. Still users with an {\tt adb} experience should feel
right at home.

The originally released version 1.0 was written by Johann Ruegg and Don Dugger.
This document describes version 1.2 prepared by Anthony Howe and
Micha{\l} Jaegermann.

This version provides multi-command input lines, which can be attached to
breakpoints and stepping commands. It gives an opportunity to record a
debugging sesion in a file, to define function keys, and other assorted
niceties. It accepts the symbol table formats of both Sozobon C (Alcyon C style)
and Mark Williams C.

If your compiler's symbol table format is not supported, or there are changes
or additions you wish to add, the source is provided free with the debugger.

\section{First steps}

\subsection{Starting}
It is possible to run \szadb\ either from a desktop or from a text shell, in
either low, medium or high resolution. In this tutorial we will assume that
\szadb\ was launched from a command line on a screen 80 characters wide (this
setting affects only the layout of some displays) and that the executable is
called \adb, in tribute to its older brother.

The simplest way of starting \szadb\ is to type something similar to
\begin{exmpl}
	\adb\ {\it program.ext}
\end{exmpl}
where {\it program.ext} is any executable.
Note that the file extension (\name{tos}, \name{ttp},\name{prg})
must be provided.
For debugging purposes, it is preferable to compile your
program with debugger information (symbol tables). For Sozobon pass the option
\name{-t} to either \name{cc} or \name{ld}.
It is possible to debug a program without this
information but this is not a trivial task. This tutorial assumes that symbolic
information is provided.

For debugging purposes, it is much preferable not to strip off symbol
tables.  In this tutorial we will always assume that this was not
done.
To achieve that effect, while compiling 
with Sozobon C, pass \name{-t} flag to \name{cc}.

If you do not have your favourite test program handy then
you may compile from the provided sources
and use for experiments a simplified version of \name{unexpand.tos};
it replaces, if possible, runs of white space 
from stdin with tabs and writes results on stdout.
Further examples will use that program compiled by Sozobon C compiler,
version $1.2,$ with \name{-t} and \name{-O} flags.

Once the debugger is started you should see on your screen a display
resembling the following:
\begin{exmpl}
	{\tt Szadb version 1.2mj+ach(english)}\\
	{\tt >} 
\end{exmpl}
indicating that everything is in order.
The character {\tt >} is the prompt.  Its presence is the most striking
visual feature showing that we are not dealing with the ``real''
\name{adb}.

\subsection{Where am I?}
If you hit at this moment \key{Return} key \szadb\ should respond
with
\begin{exmpl}
	\verb|__start:|
\end{exmpl}
The debugger positioned itself at the very beginning of a loaded program.
\szadb\ ``remebers'' the last command and its current location, which
is known as ``dot''.  You may set ``dot'' by typing any valid expression
which will evaluate to an address. Hitting \key{Return} by itself repeats
the last typed in command (not the executed one! This distinction will
be important later).

In this version the ``dot'' is set to low TPA, or a starting execution address.
The initial default command is \name{/a}
which will print the symbolic value of ``dot'' followed by a colon.
To see a numerical display of the ``dot'' try the following:
\begin{exmpl}
	{\tt .=X \key{Return}}
\end{exmpl}
for a hexadecimal address, or
\begin{exmpl}
	{\tt .=D \key{Return}}
\end{exmpl}
for the same one in a decimal form.

A note for {\sc Unix} hackers. In \szadb\ there is no distinction
{\it objectspace} and {\it dataspace}.
Therefore prefixes \name{?} and \name{/} in commands are equivalent.
Typing \name{?a} will cause the same effect as typing \name{/a}.

Addresses are printed in a form \name{symbol+offset}, where possible.
If there is no symbol table, or if \name{offset} is getting too big,
you will notice that addresses are printed as hexadecimal values.
The offset limit has initialy value of 0x400 but it can be changed by
\verb|$s| request, in the form 
\begin{exmpl}
	\verb|<new_value>$s|
\end{exmpl}
The default number base for commands and displays is sixteen (hexadecimal).
Please refer to the documentation on how to change the default base and use
numbers in other bases.

White space, which is not a part of a literal string, is not significant
in \szadb\ requests as long as a total number of characters in the command
line is below an input buffer length (78 for this version).  \key{Return}
always terminates a current line.
Try adding some blanks and tabs to previously typed commands.

\subsection{Disassembling}
Lets try something more exciting.  A request for printing machine
language instructons is \name{/i}.  It may be preceded by a count.
Try \name{,4/i}.  The comma is necessary and informs \szadb\
that what follows is a number of times to repeat the request.
If you will omit it then the debugger will decide that 
you want to start with a location 4.
If all is well then \szadb\ will respond
\readexample{firstdis.exm}
After this request a default command becomes \name{/i}.  Therefore to see
the next four locations it is enough to type \name{,4} and \szadb\
will show
\readexample{reptdis.exm}
\par
Note that the ``dot'' has moved.  This is a common feature of all memory 
examining requests. The  ``dot'' will be set past all already scanned
memory.  If you want to look at the same memory area one more time,
possibly using a different request,
use \name{\&} to reset your position.
It is a shorthand for the last {\em typed in\/} address.

\subsection{More on talking to \szadb}
A general \szadb\ command has a form
\begin{exmpl}
	{\it address ,count request\_with\_its\_modifiers}
\end{exmpl}
where each of three parts is optional or possibly not used.
Please refer to the documentation to see all available requests.
For all practical purposes count $-1$ means forever.
Read a little bit further before trying this.

Display format modifiers can be concatenated together. For example,
the following
\begin{exmpl}
	{\tt main,9/ai}
\end{exmpl}
will print the first nine instructions,
labelled by their addresses, starting from \verb|_main|. Like this:
\readexample{maindis.exm}
Note that the leading underscore, required to produce an internal form of
a symbol \name{main} was prepended automatically.  To get to an
address of \verb|__main|, if such symbol in your program exists,
you have to type its name in full.  Similar but slightly different
rules will be in force if your program has a symbol table
in the MWC format.

If you do not know which symbols are available issue a request \verb|$e|.
A display similar to the following will start to scroll accross your screen.
\readexample{symbols.exm}
The general method to stop a scrolling screen
for a moment is to use \key{\carret S} and any other key will
continue.  A \key{\carret C} will cancel the command and any further
output.

Some hexadecimal numbers may look like symbols.  For example, if you
happen to have a symbol \name{abba} in your program then \szadb\
will understand \name{main+abba} as a request for setting the ``dot''
to an address which is a sum of addresses of \name{main} and \name{abba},
even if you really meant an adress at offset of \name{0xabba} from
\name{main}.  To avoid this misinterpretation it is enough to type
\name{main+0abba} --- a number has a leading zero.
It the symbol \name{abba} is not defined then the ambiguity does not arise.
Assuming that the default base is sixteen \name{abba} will be taken
as a number. 
Otherwise such expression will be not accepted and you will see
only an error message.
The form \name{0xabba} has a unique meaning and always works.

\section{Running under \szadb}

\subsection{How to run --- with arguments}
To run a loaded program, with a name passed as a \szadb\ argument,
one has to type \name{:c}, which is a short for \name{:continue}.
When the program is running \szadb\ switches to the program screen which
is different from that one used by the debugger.
In particular, all program keyboard input will be accepted from the program
screen.  Unfortunately there is no way, at least not in this version,
to read the debugged program standard input from a file.  It has to
be typed in.
To switch from the \szadb\ screen the program window use \key{\carret W}
and return from the visit with any other character.

The program itself may have arguments.  In principle there are
two methods with which they can be set.
Firstly, you may specify them when starting \szadb.
Everything which
follows the name of a loaded executable will be taken as its arguments
and copied verbatim to its basepage. A request \verb|$p| will display the
whole basepage and it will allow you to check if you really got what
you expected.
\szadb\ does not allow for any argument extending schemes
and it will trunctate command lines which are too long.

The second method allows you to specify arguments as an optional tail
of \name{:c} request.  Once arguments were set, by any of these methods,
they cannot be changed and further attempts to do so will be ignored.

Since both shell command lines and 
the input line inside of \szadb\ are limited in length,
and partially already taken,
it may appear that there is no way to fill all available basepage
space with program arguments by any method,
short of writing directly to a computer memory.
As we will see later this is not true, even if 
it requires a little bit of trickery.
(See further descriptions how to define and execute function keys.)

When you are ready to quit, because you are done or lost and wish to start
afresh, enter \verb|$q|. If the debugg process exited of its own accord
then \szadb\ will terminate too.

\subsection{Setting breakpoints}
Executing a program under a debugger is of no great use without breakpoints.
Here is the simplest way in which they can be set.
\begin{exmpl}
	{\tt main:b}
\end{exmpl}
We can do even better than that. Try
\begin{exmpl}
	{\tt main:b ="My first szadb breakpoint"n;.=XD;,8/ai}
\end{exmpl}
Everything which follows \name{:b} on the input line will be stored and
executed later on when the breakpoint is hit.
Multiple commands are separated by semicolons.
By the way, you may use semi-colons for immediate requests too.

Breakpoints also can have counts.  Using as an example \name{unexpand.tos},
and a fragment of its disassembled code shown in a previous section, we 
may set a breakpoint
\begin{exmpl}
	{\tt main+18,3:b="about to read"n}
\end{exmpl}
in a main loop of this program, just before {\tt fgetc()} is called.
With this count an execution will stop only 
for every third character to be accepted.

Setting a breakpoint on the top of an existing one is allowed and it will
simply cause a replacement --- changing possibly a count and
commands to execute.  A list of all current breakpoints, with their counts
and associated commands,  is produced by the \verb|$b| request.
Information shown at the bottom of this list will be explained later.

It is not the best idea to set a breakpoint somewhere between two program
instructions.  Nothing terrible will happen immediately,
but your debugging run may end up prematurely amid an utter confusion.
Sometimes it is possible to
restart a wayward program by writing a needed address directly into a
program counter with a {\it address}{\tt >pc} request, but this is not
guaranteed to work.  It is also advisable to keep breakpoints on an
execution path.  They are a limited resource and there is no point
wasting it.

\subsection{Displaying information}
\szadb\ provides many ways to display information about the state
of your program.  Some of these requests were detailed above. Another
is \verb|$r|, which will show the contents of all registers
and status flags. If you are interested
in an individual register then use something like
\begin{exmpl}
	{\tt <a0=X}
\end{exmpl}
Replacing above {\tt =}
with {\tt /}, or {\tt ?}, will bring a hexadecimal display of the long
word stored at the location pointed to by register {\tt a0}.
Careful here, the last form will move the ``dot''.
There are many other possible formats.
Try, for example, \hbox{{\tt main,20/x}} and \hbox{{\tt <b,2/s}}.

The second example uses one of four read-only variables provided by
\szadb, which are
\begin{exmpl}
	\makebox[.70in][l]{\tt l}lowest text address\\
	\makebox[.70in][l]{\tt t}length of the text segment\\
	\makebox[.70in][l]{\tt b}start of the bbs segment\\
	\makebox[.70in][l]{\tt d}length of the data segment
\end{exmpl}
With an exception of {\tt l} names follow the {\sc Unix} convention.
They will be particulary handy if you will have 
a misfortune of debugging executable without a symbol table.

Formats in requests can be combined.  Let us try something like
follows.
\begin{quote}
   {\tt ="Text memory dump"2n;main,<b-main\%8+1/4x4\carret rr|rr8cn}
\end{quote}
and here are initial lines of a resulting display, where ``.''
replaces all non-printable characters.
\readexample{dump.exm}
Note that division is denoted by a \verb|%| character and that 8
divides a difference \verb|<b-main| and not only \verb|main|,
since all expressions are evaluated in strict left-to-right order.

Let's break down the format modifiers to see what is actually happening
\begin{exmpl}
  \makebox[.70in][l]{\tt 4x} print four short words in hex,\\
  \makebox[.70in][l]{\tt 4\carret} backup the ``dot'' by four
	current fields (short words),\\
  \makebox[.70in][l]{\tt rr|rr} print 2 blanks, 
       vertical bar, and 2 more blanks,\\
  \makebox[.70in][l]{\tt 8c} print 8 characters,\\
  \makebox[.70in][l]{\tt n} and a newline.
\end{exmpl}

Displays that are wider then a current screen width (40 or 80) will
have lines split automatically.

\subsection{Recording your session}
A request \verb|$>|{\sl filename\/} starts writing a transcript 
of everything which shows
on your screen to the file {\sl filename}. All examples longer than a couple of
lines were prepared this way. If the {\sl filename\/}
is missing then the currently
opened transcript will be closed. The output is always appended to the given
file, so that it is possible to open, close, and re-open the same file any
number of times.

Because GEMDOS is not re-entrant it is not a very
very good idea to perform an actual file write while processing
a GEMDOS call.
It is nearly certain you will crash your system.
The safest course in such spots is to turn recording temporarily off.
However, transcript output is buffered by default, so actual
writes occur only when the buffer is flushed when full or because the file was
closed. Therefore with proper care, one can empty the buffer prior to
dangerous spots and even create a record of a GEMDOS call, provided it is not
too wordy. A handy definition for a function key to do this is 
\hbox{\verb|$>;$>|} (see the last section on function keys).

It is advisable not to write files to your hard drive, instead use a RAM drive
or a dedicated floppy (which can be reformatted in case of disaster). Remember
that a buggy program and the debugger can write anywhere. Over system file
buffers and cached File Allocation Tables as well.

If you need all memory you can get it is possible to turn off
transcript buffering with command line option {\tt -nb}.  But then
you will have to be extremely careful about possible conflicts with
GEMDOS.

{\sc Note!}  After \verb|$>|{\it file\/} request your default command
is \verb|$>| and not the last command you were executing previously.
It is possible to execute \verb|$>| inadvertently by hitting \key{Return}
or by making some mistake while typing the next line.  This will close
your transcript with obvious results.  When something like thats
happens, or when in doubt, issue another \verb|$>|{\it file\/}.

\section{Bug hunting}

\subsection{Compiling for \szadb{}}
Lets have a closer look at the C program below.  Its stated purpose is to
replace, if possible, runs of white space with tabs.  The width of a tab
is fixed and equal to a constant \name{TABSTOP}.  If a text line is 
getting too long then substitutions are abandoned and remaining
characters are copied without modifications.
\bigskip
\listing{\exmpskip}{unexpand.c}
\bigskip
\noindent
This program has actually two bugs.
See if you can find them just examinig the source.

Here is how \szadb\ can help. Compile source as follows
(these commands are for Sozobon~C)
\begin{exmpl}
	{\tt cc -t -O -o unexpand.tos unexpand.c}
\end{exmpl}
Flag \name{-O} is not necessary but with this particular compiler
you will probably find a disassembled code easier to follow.
You may test the compiled executable on its own source.
\begin{exmpl}
	{\tt unexpand.tos <unexpand.c >output }
\end{exmpl}
In the first moment the program appears to work, but a closer examination
of the \name{output} reveals that the indentation is not exactly right.
Moreover, some lines start with a blank, followed by a tab, which is
not really what was intended.
There are also other problems.  Check yourself.
The likely suspect will be an array \name{tabs} of tabstops filled by
a function \name{settab()}.

\subsection{The first bug}
Start the program under \szadb\ control
\begin{exmpl}
	{\tt adb.ttp unexpand.tos}
\end{exmpl}
and set a breakpoint at \verb|settab+4|, just after {\tt link} instruction.
Run the program with \verb|:c ; $C|. This will produce the following
display
\readexample{bcktrace.exm}
In the absence of better information all arguments 
shown in the stack backtrace are assumed to be two bytes wide.  
We know from the source that \name{settab()} actualy
expects one pointer.
Confirm that it got a right one by putting it together from two halfs
and using the request \verb|68bb4=p| to print it as the symbol.
You should see \verb|_tabs| in response.
To continue execution of the current function 
till it returns to its caller, use {\tt :f} which stands for {\tt :finish}.

It is clear from lines 30 and 35 that the first tabstop in
\verb|_tabs| is expected to be on a position \hbox{{\tt TABSTOP - 1}}.
Dumping some initial fragment of the just initialized array 
we see the following:
\readexample{tabs.exm}
This is clearly wrong and one bug becomes obvious.
To repair it line 59 should be changed to
\begin{exmpl}
	\verb|for (i = 1; i <= MAXLIN; i++) {...}|
\end{exmpl}

\subsection{\ldots{}and the other one}

The second bug is harder to spot, since for most of test inputs our
program will work correctly.
This is a typical example of a program broken for boundary conditions.
To make it easier to track the problem set
\name{MAXLIN} to some small integer (around 10 should be good),
recompile the program, restart the debugging session and set a breakpoint
at \verb|main+28|, just after a character was read.
Set this breakpoint with a count and a request 
to show a received character with
\begin{exmpl}
	{\tt main+28,5:b <d0=cx}
\end{exmpl}
With carefuly chosen input this will show where you are in the
program and will skip unnecessary stops in the same time.
You have to provide an input by typing it yourself.
Remember that to repeat the last command  {\tt :c} it is enough to hit
\key{Return}.
For execution defaults \szadb\ will use the most recently typed command
even if some other requests were executed by breakpoints.

In order to see how the received character is processed you can single step
with \name{:s}, which will follow program execution.
The request \name{:n} will single step like \name{:s} but will
execute function calls at full speed and so have the
effect of stepping over them. Breakpoints set on skipped
levels still will be obeyed.

Tracing some tests inputs with the value of the variable \name{col} around
\name{MAXLIN} should reveal the second error soon enough.  If you
want to try it yourself, do not read further.

Looking at line 17 we find a sharp less-than inequality in the test. It should
be replaced by a less-than-equal-to. Otherwise, when the value of col
equals MAXLIN, and your input is "right", the program is trying to read, in
lines 28 and 31, from the location \verb|&tabs[col]|,
which is one past the end of the array.
The remaining analysis of this bug is left as an exercise to the reader.


\subsection{Breaking out}

Another stepping command is the \name{:j} jump request.  Its purpose
it to short-circuit loops.  It behaves exactly like \name{:n} in that
it skips-over function calls but unlike the \name{:n} request which
follows branch instructions, the \name{:j} will place a temporary
breakpoint at the instruction immediately following the current
instruction in memory.  Think of \name{:n} as step-next-logical
instruction and \name{:j} as step-next-physical instruction.  The idea
is that you only use \name{:j} to step-out of loops where you are
sitting on the loop-back branch.
\readexample{jumpone.exm}
In the example above, if you are sitting on the \name{bnz} instruction and you
wish to execute the remainder of the loop at full speed then you use the
\name{:j}, which places a temporary breakpoint at the location \name{.end}.
However care
must be taken when using this command because it is possible that the 
flow-of-control never reaches the temporary breakpoint.
\readexample{jumptwo.exm}
This is a case where NOT to use the \name{:j} command. 
Basically to use the \name{:j}
request you should know how your C compiler sets up its loops or where your
assembler code is meant to go. It is sometime a good idea to put a safety
breakpoint somewhere you know you will end up.

Because of the unusual nature of the \name{:j} request,
it will not autorepeat if the next command is just a \key{return} key. 
Instead \name{:n} is performed. Also it is
also considered a variant of \name{:n} when attaching execution requests to
stepping commands (see further down).

It should be also mentioned that all stepping commands have upper case
counterparts, which are noisier. When used they return the a full register
display after the step (\name{:S} is like doing \verb|:s;$r|).

\section{More fun and games}

This section covers some finer points of \szadb\ use.
You may put them aside when just starting first experiments
with the debugger.
But probably one day you will find some of that information very
useful.

\subsection{Advanced steps}

It was already mentioned that all stepping commands may have also
attached \szadb\ requests.  As a matter of fact there is even one
default, for \name{:f}.  To get a similar effect for other steps
just type what you want to be executed after a given command on
the \szadb\ input line. For example
\begin{exmpl}
	{\tt :s ="this string printed by :s command"n}
\end{exmpl}
You will notice that this form of \name{:step} does not
move you ahead in the program.
This gives an opportunity to set and modify requests in advance,
without an immediate execution.
It also saves your nerves if you are not a very good typist.
To really step by one instruction forward hit \key{Return} and
observe what will happen.
As noted before \name{:j} is a variant of \name{:n} and setting
request for it will really change what is attached to \name{:n}.

The example above is not tremendously useful,
but tracking values of a chosen registers can be.
Or anything else that you need and that will fit on the command line.
This condition is much less limiting than it appears
when used in conjunction with function keys.

There are two points to remember.
There is no syntax check while you are setting requests.
Watch what you are typing if you do not want to see only error
messages.
And there is nothing to prevent a direct or indirect recursion.
Since a depth of \szadb\ stack is limited and \key{\carret C} does
not always work it is better to avoid such constructs.

If you came to a conclusion that the only way to
change requests attached to a stepping command is to
overtype them with something else then you are correct.
Moreover an empty string is not a good replacement 
since it causes an execution.
Luckily there is some other way to turn a noise off.
Try typing \verb|::s-| and similar commands for \name{:n} and
\name{:f}.  Switched off requests are not gone, unless later redefined.
They can be brought back by a similar command as above in which 
\verb|-| was replaced with \verb|+|.
The default request for \name{:f} is special.
It cannot be overtyped or turned on by \verb|::f+|.
To bring it back use \verb|::f`| instead.

There is one more way of request switching.  If you will type,
for example, \verb|::n_|, then whatever was attached to  \name{:s}
will be performed also for \name{:n}, instead of a ``native'' command.
This gives an opportunity to create, say, a quite complicated
request for \name{:n} and most of the time execute a simpler
request for \name{:s}.
When an original request for \name{:n} is
needed it is enough to type \verb|::n+| to get it back.
The same mechanism works for \name{:f}, temporarily redirecting
its requests ``down'' to the first active one.
There is no similar switching going the other way.
To gain a better understanding try all of this after defining
various requests for each of stepping commands and examine
effects with the \verb|$b| command after every switch.

It is also possible to make user breakpoint silent or not. For example
\begin{exmpl}
	{\tt main:b ="entering main"}
\end{exmpl}
{\tt main::b-} and {\tt main::b+} turns the action off and on.

\subsection{How to use function keys}


Function keys can be defined, only once per debugging session, by reading
their definitions from a file. The name of the file can be passed on the
command line with \name{-k} {\sl filename}. The file may be as simple as this
\readexample{kdefs.exm}
Note that an uppercase F, which starts key definition, has to
be in the first column and it has to be immediately followed
by a valid key number.  Shifed functions keys have numbers
between 11 and 20.  A definition of \key{F2} is continued on the
next line since a terminating newline is escaped with a \verb|\|
character.  Otherwise this continuation line would be simply
ignored.  \szadb\ also does not pay any attention to the second
definition of \key{F1}.  To see a list of all function keys defined
type \verb|$k|.

Defined function keys can be used in two ways.  Just hitting
a function key will insert into the current input line as many
characters from a corresponding string as it will fit.
The resulting input text can be edited.
In order to execute a full definition use \verb|$k| followed by a key
number --- always in decimal.
For example, a request \verb|$k2| will print twice
the message, which was just defined in the function key file.

Note that there no way to specify that a function key should execute
immediately instead of waiting for a \key{return}. Also is there no default
function key file name 
like --- \name{adb.key} --- so you'll have to specify the 
\name{-k}~{\sl file\/}
option on the command line each time or use a command alias.
Note also that requests of a form \verb|$k|{\sl $<$number$>$\/} do not
autorepeat.

It is clear that strings of commands attached to function keys
can be longer than a lenght of the input line.  Actually around
2K will be accepted.  This is hopefuly longer than any \szadb\
script you ever want to write.  Especially if you take into
account that one script can call another.  Warnings against
recursion apply as well.

Such long scripts can be, of course, attached to breakpoints
or stepping commands.  They give also, previously mentioned,
opportunity to fill all available space on a base page with
command line arguments.  Just define one of function keys as
\begin{exmpl}
	{\tt :c <text to put on a base page>}
\end{exmpl}
and execute at the beggining of your session.  Coupled with
a possibility of directly modifying the memory this allows 
for a preparation of scripts which will emulate any extended arguments
scheme, even if \szadb\ directly does not support directly any
of these.

\subsection{Other symbol table formats}

This debugger was designed as a companion for Sozobon~C and therefore
it understands its symbol table format, which was inherited from
Alcyon compiler.
The version for which this document was written also supports MWC ---
currently the official Atari development compiler.
It will cooperate also  with the ST version of GNU compiler gcc,
if you will decide to get your hands dirty in a source code.

\subsubsection{Mark Williams C support}

Symbols created by MWC are outwardly different from those produced
by Sozobon C in that that names can be longer  --- up to
sixteen characters --- and an underscore character is appended instead of
beeing prepended.
When \szadb\ will detect an MWC produced object it will apply its
conventions.
That means that if you type \name{main} it will first try
to find a symbol \name{main}.
If this fails it will search for \verb|main_| next  (not for \verb|_main|).
If a symbol name has a leading underscore, or more
than one trailing, you have to type them yourself.

The debugger is trying to guess by itself which compiler produced
the current executable.  If it guesses wrong you may always override
its choice by dropping a hint on the command line.
It consists of an \name{-os} flag for the Sozobon format and \name{-om} for MWC.
Remember that if you work in an assembler the guessing code can always
be fooled.  The flags always provide a way to set things straight.

When writing this guessing code I had no official description
of the format used by MWC.  All necessary information was inferred
from an examination of MWC produced binaries.
The code  worked so far on everything
I tried, but it may happen to be wrong for your version.
Since \szadb\ comes complete with source you can
modify it accordingly and recompile
(look for all places where a global variable \name{swidth} is modified).

\subsubsection{What to do with gcc}

A default form of executables produced by this compiler is with a symbol
table attached. You need to use \name{-s} flag if you really do not want
it produced.  Moreover, a TOS version is using the same symbol format
as Sozobon C with one subtle difference.  A bit which carries an information
that a symbol is global, known as \verb|S_EXT| in \szadb\ parlance,
is not set.  This makes the debugger blind to a symbol presence.

Here are instructions how to modify file \name{ld.c} which contains sources
for gcc linker. In a function \verb|write_atari_sym(p, str)| for TOS version
add the following
\begin{exmpl}
	\verb? if (p->n_type & N_EXT)?\\
	\makebox[1.5cm]{}\verb? sym.a_type |= A_GLOBL;?
\end{exmpl}
just before a line which reads
\begin{exmpl}
	\verb? sym.a_value = p->n_value;?
\end{exmpl}
Recompile and reinstall linker and from this moment on \szadb\
recognizes gcc produced symbols.

If you do not have these sources or you cannot recompile them ---
because you do not have enough memory, for example --- not everyting is
lost.  It is quite feasible to disable \verb|S_EXT| check in \name{setsym()}
(look in the file \name{adb1.c}).  
You will not notice any change for Sozobon C created executables.
All symbols occuring there are actually always global.
This is not quite true for gcc and some new, sometimes strange, symbols
will appear but usually this will not create any problems.
In order to have only globals in a gcc produced symbol table
pass \name{-x} flag either to gcc or to its linker.

\subsubsection{Other compilers}

If you own a compiler which produces symbolic information in an
unsupported format you can modify \szadb\ yourself to support it.  For
a model of how to do this look for \name{setsym()} and
\name{mwsetsym()} in the file \name{adb1.c}.  Both of them call
\name{addsym()} which performs an actual insertion of a new symbol and
its value into a linked list of supported symbols.  A current version
of \name{addsym()} will handle symbol names of any length but internally they
will be chopped off to something not longer than the current value of
the global variable \name{swidth}.  Ensure that this value is set properly
for your needs.

The mechanism above can be easily extended to allow reading some
symbols and their values from a user suplied text file.  Some may find
it very handy.  The current version of \szadb\ does not support this
feature.  It is possible to roll your own if you really need it.


\subsection{Customization}

Some possible customizations were already mentioned.
The other obvious one is a version of  \szadb\ without on-line help.
If you feel brave enough remove the definition of 
the compile time constant \name{HELP} from the \name{makefile}.
A size of the executables will undoubtely go down.

In order to create a version in some other language edit 
a file \name{lang.h}.
It contains all messages which \szadb\ may display.

You can adjust to your taste some other points.  For example,
what are default requests attached to stepping
commands and what is their status (look in \name{stepping.c} for
\name{bpstat()}, \name{findcmds()}, \verb|bpt_list[]|).
Which characters are used for requests switching
(see \name{getrequs()} in \name{pcs.c}).
Some features, like support for functions keys or other symbol
table formats, can be taken out easily without affecting the
whole design.

If you try to recreate \szadb\ with another compiler you should note
that essential parts of this program were written in assembler
and may have to be translated to something your tools can accept.
Here is another point to watch for.
A variable \verb|_BLKSIZ| sets size of \name{Malloc}'ed blocks.
If a library you use supports something similar modify
accordingly.
There are also some constructs in C code,
like a static initialization of a union member (\verb|bpt_list|
in \name{stepping.c}) and an array of size 0 in a definition
of \name{struct symbol} in \name{adb.h}, which 
are accepted by Szozobon C but can give
a hiccup to some other compilers.
These points can be modified without undue strain.
Otherwise the code should be pretty portable, although it is
ST specific by its very nature.

\subsection{Running on a verge}

You may find yourself in a situation when your program wants all
memory it can get and together with the debugger does not exactly
fits into available space.
There are still some things which can be done.

Keep in mind that \szadb\ grabs all memory it needs before a program
to debug is loaded and it does not make any claims later.
These requirements can be minimized. 
Function key definitions if non-existent will not use memory at all,
discounting the supporting code.  
This code by itself is mostly contained in a file \name{fkeydefs.c}
and it is easy to remove.
The flag \name{-nc}, for {\tt no commands},  gives some memory for a price
of missing breakpoint requests.
A more substantial memory chunk can be released with 
\name{-nb}, for {\tt no buffering} on transcript, flag.
You still can open a transcript file but all output will be direct.
Beware of GEMDOS when using transcripts without buffering.

If this does not help then there is a time to trim some fat from
\szadb\ itself. You will have to recompile it leaving some features
out.  On-line help is probably the first candidate.

All this effort can be wasted if you will forget about one thing.
Due to an infamous design bug in TOS a Malloc system call can
be invoked only a small fixed number of times. When this pool is
exhausted you will get ``out of memory condition'' even if memory
is still plentiful.
Therefore all allocation functions, from smart libraries,
request a memory from the system in bigger pieces and later
try to satisfy all requests chopping from an already owned resource.
The name of the game for \szadb\ is to use for all its needs
only one chunk of a system memory which is just big enough, so
not too much of an unused memory will be left.
The whole symbol table of a de\-bug\-ged program must fit there and
all \szadb\ requests for a space for in\-ter\-nal structures have to
be statisfied.
To adjust sizes properly
you may want to change a constant \name{CHUNK} which is defined
at the top of a file \name{adb.c}.

If everything else fail you may still try to change a value of
a global variable \verb|__STKSIZ| from \name{start.s} but this
would be probably the last stand.

\subsection{Writing to memory}

It can be done.
This is left as an exercise to the reader.
Check your documentation.

%}   % turn this on if you want to make life easy for dvi2tty
\end{document}
