@comment Tell Emacs to use -*-texinfo-*- mode
@comment $Id: intro.tex,v 2.2 91/09/01 23:04:29 royce Exp $

@node Introduction, Fifteen Minute Guide, Copying Conditions, Top
@unnumbered Introduction

Fnordadel is a bulletin board system (``@sc{bbs}'') program.  It is a member of
the family of @sc{bbs} programs descended from Citadel, which was written by CrT
in the early 1980's.  Fnordadel currently runs on the Atari ST and TT
machines only.  There are many other variants and clones of Citadel available
for a wide variety of machines, including the IBM PC and its clones, the Mac,
the Amiga, Unix machines, and others.  Fnordadel is derived from
STadel by David Parsons (orc).  @xref{History}, for more on the
history and lineage of Fnordadel.

Citadels operate on the ``room'' metaphor, where discussions take place in
named rooms which can be easily and dynamically managed.  Citadels are thus
considerably different from most other @sc{bbs}es, which tend to use more static
``message areas'' and the like.  Citadel is intended for discussion systems,
not heavy file-transfer sites; however, reasonable file transfer mechanisms are
provided.  The user interface is orthogonal, fast and highly interactive, while
shunning the typical menu-based interface of most @sc{bbs} programs.

A few significant features of Fnordadel (and, in general, all Citadels) are:
@itemize @bullet
@item
@dfn{Ease of maintenence}. Fnordadel requires little human intervention to
operate.  Most system files are fixed-size circular files which are inherently
self-maintaining.  Barring power failures and bugs, you can let a
Fnordadel run for months without even looking at it.
@item
@dfn{Ease of configuration}.  All configuration options are in one
well-documented file.  @xref{Fifteen Minute Guide, , Fifteen Minute Guide
To Fnordadel}. 
@item
@dfn{Runs on any ST}.  Even if you have only a 512k machine with a single
floppy drive, you can still run Fnordadel.
@item
@dfn{Good connectivity}.  Fnordadel supports standard Citadel networking
with minimal incompatibilities with other Citadel variants.  You can share
mail, rooms and files with other Citadels.
@item
@dfn{Flexible security}.  Citadels are traditionally open systems, but if you
have pests, Fnordadel has a fair number of ways to help control them without
sacrificing too much of the traditional openness.
@end itemize

Oh, and since you may have been wondering @dots{} : ``fnord'' is a subliminal
trigger word, adapted from a well-known science fiction work.  Since we figure
hacking on @sc{bbs}es is about as subliminal as you can get, it fits.  Besides, 
all the good names were taken.

@node Notation,  , Introduction
@unnumberedsec Notations Used in This Manual

Several standard notations are used throughout this manual, the online help
files and in general Citadel usage which you should be aware of.

@enumerate
@item
The sequence @samp{^X}, where @samp{X} is a letter or symbol on the keyboard,
represents a control character.  To type one, hold down the @samp{Control} key
and press @samp{X}.
@item
The sequence @samp{<FOO>} refers to one of the special keys on the keyboard
such as the carriage-return key (@samp{<CR>}), the escape key (@samp{<ESC>}),
etc.  This notation is also used occasionally to refer to special characters
which may not have their own keys on the keyboard; an example is the linefeed
character, @samp{<LF>}.
@item
The sequence @samp{[X]} where @samp{X} is any key, indicates a Fnordadel
single-key command.  Single-key commands are usually referenced using a full
word or two, indicating what the command means---usually this is derived from
the words that Fnordadel echoes back to you when you press the key.  For
example, @samp{[N]ew} refers to the command to read new messages; you
press @samp{N} and the system echoes @samp{New}.
@item
The sequence @samp{.X(stuff) Y(stuff) Z(stuff) @dots{}} indicates a Fnordadel
multi-key command.  Such commands always start with a special character such
as @samp{.}, @samp{;} or @samp{!}.  The text in parentheses represents what
Fnordadel will echo back to you as you type the command.  For example, the
notation @samp{.R(ead) X(modem) N(ew)} means that you type @samp{.RXN} and
the system will echo @samp{.Read Xmodem New}.
@end enumerate

Please note that case is insignificant; i.e., a capital letter is the same as a
lower-case letter so far as commands are concerned.

@node A Warning
@unnumberedsec A Warning

This manual makes occasional use of @emph{humor}.  Read with caution, as we
wash our hands of any responsibility for any offense generated in
unappreciative or humorless readers.
