@comment Tell Emacs to use -*-texinfo-*- mode
@comment $Id: support.tex,v 2.3 91/09/02 01:37:58 adrian Exp $

@node Fnordadel Support, History, Miscellaneous, Top
@appendix Fnordadel Support
@cindex Support

If you have not yet read the file @file{COPYING}, or read the copying
conditions section of this manual (@pxref{Copying Conditions}), please do
so.  In a nutshell, though, it says that Fnordadel is completely
uncopyrighted, and in the public domain.
Therefore, we cannot charge money for it.  (If you feel like donating
to the ``Supply Adrian and Royce With Coca-Cola'' Fund, feel free to do
so---but we just want to make clear that you won't get anything for it besides
our undying gratitude, and perhaps the bottle deposit refunds.)

Support, then, is free.  If you're obnoxious, you won't get any
support; if you are, like most people, reasonable and patient, you'll get
all the support we can give you.

The sort of stuff we'd like to hear about:

@itemize @bullet
@item
Bugs.  (Yes, they do happen@dots{}).  The first thing to do when you run
across what you think is a bug is to check the file @file{bugs.doc} to see if
it is already known.  Next is to try re-reading the documentation to see if
it's meant to be that way.  If this fails, report it as a bug.  Try to give
as much detail as possible; reports like ``Help! Networking is buggered!''
don't give us much to go on.  Details of the results of the bug are OK.  Hints
on what you did to cause the weird behaviour are good.  Detailed directions
on how to reproduce the bug are The Best.

@item
Incorrect documentation.  If you ever locate something in the
copious Fnordadel documentation that just isn't true, tell
us so we can fix it.  We try to proof-read to some degree, but
our calling is to write code, not manuals.  The latter we do
only under protest.

@item
Problems with getting it to run.  Please try to ask around
locally, first, if there's anyone else in your area running
Fnordadel; if that fails, send us mail or post in the
@code{FnordTech)} room, and we'll help.  If
your difficulties point out short-comings in the documentation,
please let us know so we can make some additions.

@item
Inefficiencies/annoyances/etc.  If something doesn't work the
way you think it should, or you can think of a way to make it
more useful, or whatever, we want to hear these things.  There
are some things in Fnordadel that we never use, and that
will not have attracted our attention; there are other things
which have been nagging at us for a while, too, which we've
never got around to fixing.  Maybe if we're cajoled a bit@dots{}.

@item
Suggestions for new features.  Is there something you've been
dying to see added to Citadel?  Let us know; we make no
promises, but we do our best.
@end itemize

We can be reached directly at our systems:

@display
Adrian (@code{elim@@secret}, CA 403 425 1779, 2400 baud)
Royce (@code{Mr. Neutron@@RT}, CA 403 455 2709, 2400 baud)
@end display

We are reachable by Citadel mail via:

@example
@dots{}!C-86 Test System!secret!elim
@dots{}!C-86 Test System!RT!Mr. Neutron
@end example

Even better, there is an internationally networked room called
@code{FnordTech)} devoted to
Fnordadel concerns.  To arrange a feed, ask the Sysop of your nearest
Fnordadel system, or send mail to @code{elim@@secret}.  Please note that
the room is intended to be fairly private, i.e. Sysops only, in general.  If
you want to make it public on your system, either make the room read-only,
or police
it strictly.  We've seen other system support rooms go to the dogs because
every Tom, Dick and Len was let in.  We don't want that to happen here.  If
users at large would like a place to chatter about Fnordadel or whatever,
a different room can be set up for that.

The latest Fnordadel releases are always available from the room
@code{Fnordadel dist]} on secret and RT.  Since Fnordadel allows file requests
from unknown systems, you may add either system to your net list and simply
use the
standard file request command to retrieve what you want, without having
to tell us about it first.  Or, you can phone up and download it manually,
since the room is public access.  Before you do either of these, check your
local sources first; somebody may have already gotten the latest and be able
to save you a lot of expense.

The release archives are prepared using Zoo v2.1.  We used to distribute
the archives in LHarc format, but got @emph{real} tired of all the buggy,
incompatible versions of LHarc floating around.  When Zoo v2.1 came out,
offering compression ratios equal to LHarc, we switched.  Zoo is extra
useful to us because we do a lot of work under Unix, and need to swap
archives back and forth freely between it and @sc{tos}.  There's only one
version of Zoo, the source code is freely available, and it runs on everything.

Anyway, enough preaching.  The archives are named as follows:

@table @file
@item fn@var{xyz}bin.zoo
The complete release (binaries, man-pages, helpfiles,
everything) for version @var{x.yz}.  It is split into subdirectories for
easier manipulation.
@item fn@var{xyz}man.zoo
The ASCII version of the Reference Manual, split into separate files
for easier reading and/or printing.
@item fn@var{xyz}src.zoo
A huge monster archive containing the full source code for the Fnordadel
programs and documentation.  This is not for the faint of heart.
@end table

In addition, individual program updates will be posted from time to time;
these will be individually handled.  Updates will be distributed as archives
of new binaries, and also as diffs to the sources.  Diffs will be named
@file{@var{xxxyy}-@var{zz}.zoo}; this monstrous looking naming scheme
represents an archive of diffs from version @var{x.xx-yy} to @var{x.xx-zz}.

Note that the LHarc archives may be posted in addition to the Zoo archives,
if we have an excess of hard disk space and generosity (a rare combination).
The LHarc program in use could be anything; there are so many to choose from!
It is known that certain LHarc programs will not extract from LZHes
created with other LHarc programs.  (The only standard thing about
LHarc programs on the ST is their incompatibility.)

If you run into trouble extracting either the Zoo or LHarc archives, the
@code{Fnordadel dist]} room also contains the current versions of Zoo and
LHarc, as used to prepare the Fnordadel archives.

Fnordadel may also be obtained by way of the Post.  The mailing
address is as follows:

@display
Royce Howland
12908 - 119 Avenue
Edmonton, Alberta
CANADA  T5L 2N3
@end display

Please send a stamped, self-addressed disk mailer (or enough money
to cover postage if you're in the USA or some other place that isn't Canada)
and a disk (two disks, if you've only got a single-sided drive).

So, that's it.  Even if you have no bugs to report, and nothing new
to suggest, drop us a line anyway.  We'd love to hear from you.

@display
Adrian Ashley (@code{elim@@secret})
Royce Howland (@code{Mr. Neutron@@RT})
@end display
